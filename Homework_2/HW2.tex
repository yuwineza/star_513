% Options for packages loaded elsewhere
\PassOptionsToPackage{unicode}{hyperref}
\PassOptionsToPackage{hyphens}{url}
%
\documentclass[
]{article}
\usepackage{amsmath,amssymb}
\usepackage{iftex}
\ifPDFTeX
  \usepackage[T1]{fontenc}
  \usepackage[utf8]{inputenc}
  \usepackage{textcomp} % provide euro and other symbols
\else % if luatex or xetex
  \usepackage{unicode-math} % this also loads fontspec
  \defaultfontfeatures{Scale=MatchLowercase}
  \defaultfontfeatures[\rmfamily]{Ligatures=TeX,Scale=1}
\fi
\usepackage{lmodern}
\ifPDFTeX\else
  % xetex/luatex font selection
\fi
% Use upquote if available, for straight quotes in verbatim environments
\IfFileExists{upquote.sty}{\usepackage{upquote}}{}
\IfFileExists{microtype.sty}{% use microtype if available
  \usepackage[]{microtype}
  \UseMicrotypeSet[protrusion]{basicmath} % disable protrusion for tt fonts
}{}
\makeatletter
\@ifundefined{KOMAClassName}{% if non-KOMA class
  \IfFileExists{parskip.sty}{%
    \usepackage{parskip}
  }{% else
    \setlength{\parindent}{0pt}
    \setlength{\parskip}{6pt plus 2pt minus 1pt}}
}{% if KOMA class
  \KOMAoptions{parskip=half}}
\makeatother
\usepackage{xcolor}
\usepackage[margin=1in]{geometry}
\usepackage{color}
\usepackage{fancyvrb}
\newcommand{\VerbBar}{|}
\newcommand{\VERB}{\Verb[commandchars=\\\{\}]}
\DefineVerbatimEnvironment{Highlighting}{Verbatim}{commandchars=\\\{\}}
% Add ',fontsize=\small' for more characters per line
\usepackage{framed}
\definecolor{shadecolor}{RGB}{248,248,248}
\newenvironment{Shaded}{\begin{snugshade}}{\end{snugshade}}
\newcommand{\AlertTok}[1]{\textcolor[rgb]{0.94,0.16,0.16}{#1}}
\newcommand{\AnnotationTok}[1]{\textcolor[rgb]{0.56,0.35,0.01}{\textbf{\textit{#1}}}}
\newcommand{\AttributeTok}[1]{\textcolor[rgb]{0.13,0.29,0.53}{#1}}
\newcommand{\BaseNTok}[1]{\textcolor[rgb]{0.00,0.00,0.81}{#1}}
\newcommand{\BuiltInTok}[1]{#1}
\newcommand{\CharTok}[1]{\textcolor[rgb]{0.31,0.60,0.02}{#1}}
\newcommand{\CommentTok}[1]{\textcolor[rgb]{0.56,0.35,0.01}{\textit{#1}}}
\newcommand{\CommentVarTok}[1]{\textcolor[rgb]{0.56,0.35,0.01}{\textbf{\textit{#1}}}}
\newcommand{\ConstantTok}[1]{\textcolor[rgb]{0.56,0.35,0.01}{#1}}
\newcommand{\ControlFlowTok}[1]{\textcolor[rgb]{0.13,0.29,0.53}{\textbf{#1}}}
\newcommand{\DataTypeTok}[1]{\textcolor[rgb]{0.13,0.29,0.53}{#1}}
\newcommand{\DecValTok}[1]{\textcolor[rgb]{0.00,0.00,0.81}{#1}}
\newcommand{\DocumentationTok}[1]{\textcolor[rgb]{0.56,0.35,0.01}{\textbf{\textit{#1}}}}
\newcommand{\ErrorTok}[1]{\textcolor[rgb]{0.64,0.00,0.00}{\textbf{#1}}}
\newcommand{\ExtensionTok}[1]{#1}
\newcommand{\FloatTok}[1]{\textcolor[rgb]{0.00,0.00,0.81}{#1}}
\newcommand{\FunctionTok}[1]{\textcolor[rgb]{0.13,0.29,0.53}{\textbf{#1}}}
\newcommand{\ImportTok}[1]{#1}
\newcommand{\InformationTok}[1]{\textcolor[rgb]{0.56,0.35,0.01}{\textbf{\textit{#1}}}}
\newcommand{\KeywordTok}[1]{\textcolor[rgb]{0.13,0.29,0.53}{\textbf{#1}}}
\newcommand{\NormalTok}[1]{#1}
\newcommand{\OperatorTok}[1]{\textcolor[rgb]{0.81,0.36,0.00}{\textbf{#1}}}
\newcommand{\OtherTok}[1]{\textcolor[rgb]{0.56,0.35,0.01}{#1}}
\newcommand{\PreprocessorTok}[1]{\textcolor[rgb]{0.56,0.35,0.01}{\textit{#1}}}
\newcommand{\RegionMarkerTok}[1]{#1}
\newcommand{\SpecialCharTok}[1]{\textcolor[rgb]{0.81,0.36,0.00}{\textbf{#1}}}
\newcommand{\SpecialStringTok}[1]{\textcolor[rgb]{0.31,0.60,0.02}{#1}}
\newcommand{\StringTok}[1]{\textcolor[rgb]{0.31,0.60,0.02}{#1}}
\newcommand{\VariableTok}[1]{\textcolor[rgb]{0.00,0.00,0.00}{#1}}
\newcommand{\VerbatimStringTok}[1]{\textcolor[rgb]{0.31,0.60,0.02}{#1}}
\newcommand{\WarningTok}[1]{\textcolor[rgb]{0.56,0.35,0.01}{\textbf{\textit{#1}}}}
\usepackage{graphicx}
\makeatletter
\def\maxwidth{\ifdim\Gin@nat@width>\linewidth\linewidth\else\Gin@nat@width\fi}
\def\maxheight{\ifdim\Gin@nat@height>\textheight\textheight\else\Gin@nat@height\fi}
\makeatother
% Scale images if necessary, so that they will not overflow the page
% margins by default, and it is still possible to overwrite the defaults
% using explicit options in \includegraphics[width, height, ...]{}
\setkeys{Gin}{width=\maxwidth,height=\maxheight,keepaspectratio}
% Set default figure placement to htbp
\makeatletter
\def\fps@figure{htbp}
\makeatother
\setlength{\emergencystretch}{3em} % prevent overfull lines
\providecommand{\tightlist}{%
  \setlength{\itemsep}{0pt}\setlength{\parskip}{0pt}}
\setcounter{secnumdepth}{-\maxdimen} % remove section numbering
\ifLuaTeX
  \usepackage{selnolig}  % disable illegal ligatures
\fi
\IfFileExists{bookmark.sty}{\usepackage{bookmark}}{\usepackage{hyperref}}
\IfFileExists{xurl.sty}{\usepackage{xurl}}{} % add URL line breaks if available
\urlstyle{same}
\hypersetup{
  pdftitle={STAR 513: HW 2},
  pdfauthor={YOUR NAME HERE},
  hidelinks,
  pdfcreator={LaTeX via pandoc}}

\title{STAR 513: HW 2}
\author{YOUR NAME HERE}
\date{}

\begin{document}
\maketitle

Total points: 24\\
Questions are worth \textbf{2 pts} each, except where noted.\\
See Canvas calendar for due date.\\

Homework should be submitted as a pdf, doc or docx file via Canvas.\\
Use of R markdown HW template is strongly encouraged.\\
Add or delete code chunks as needed.\\
Knit frequently to avoid last minute problems!\\
Your submitted assignment should be neatly formatted and organized.\\

\textbf{Ott \& Longnecker Example 11.32:} A chemist is interested in the
association between weight loss in lbs (y) versus the exposure time in
hours (x) for a particular compound. The data includes n = 12
observations. The data ex11-32.csv is available from Canvas.

This assignment is very similar to Lec02\_examples: Simple Linear
Regression!

\hypertarget{q1-4-pts}{%
\subsection{Q1 (4 pts)}\label{q1-4-pts}}

For this question, please use the ggplot2 package (available through
tidyverse). You may need to install this package if it is the first time
you have used it. Create a scatterplot of the data with fitted
regression line overlaid. Your plot should include axis labels that
include the units for each variable.

\hypertarget{q2}{%
\subsection{Q2}\label{q2}}

Fit an appropriate regression model and show the summary() output.

\begin{center}\rule{0.5\linewidth}{0.5pt}\end{center}

\begin{center}\rule{0.5\linewidth}{0.5pt}\end{center}

\hypertarget{q3}{%
\subsection{Q3}\label{q3}}

For this question, please use the tidy() function from the broom
package. You may need to install this package if it is the first time
you have used it. From the model you fit in the previous question,
present ``tidy'' results.

\begin{center}\rule{0.5\linewidth}{0.5pt}\end{center}

\begin{center}\rule{0.5\linewidth}{0.5pt}\end{center}

\hypertarget{q4-4-pts}{%
\subsection{Q4 (4 pts)}\label{q4-4-pts}}

Provide a detailed interpretation of the estimated \textbf{slope} in
context of this research study. Your interpretation should include
appropriate units and the numeric value for the estimated slope.

\begin{center}\rule{0.5\linewidth}{0.5pt}\end{center}

Response

\begin{center}\rule{0.5\linewidth}{0.5pt}\end{center}

\hypertarget{q5}{%
\subsection{Q5}\label{q5}}

Consider the p-value corresponding to ExposureTime. State the null
hypothesis using standard greek letter notation and subscripting. Hint:
See the end of the Lec1\_notes for LaTex code examples.

\begin{center}\rule{0.5\linewidth}{0.5pt}\end{center}

Response

\begin{center}\rule{0.5\linewidth}{0.5pt}\end{center}

\hypertarget{q6}{%
\subsection{Q6}\label{q6}}

Do we have evidence (at the alpha = 0.05 level) of a linear association
between weight loss and exposure time? Is the association positive or
negative? Justify your response using an appropriate p-value.

\begin{center}\rule{0.5\linewidth}{0.5pt}\end{center}

Response

\begin{center}\rule{0.5\linewidth}{0.5pt}\end{center}

\hypertarget{q7}{%
\subsection{Q7}\label{q7}}

Create the plots of (1) residuals vs fitted values and (2) qqplot of
residuals.

\hypertarget{q8-4-pts}{%
\subsection{Q8 (4 pts)}\label{q8-4-pts}}

The four assumptions of simple linear regression are listed below. For
each assumption, state a graph that can be used to check the assumption.
If an assumption cannot be checked graphically, write ``Cannot be
checked graphically''. You do NOT need to evaluate the assumptions for
this question.

\begin{center}\rule{0.5\linewidth}{0.5pt}\end{center}

Independence:\\
Equal variance:\\
Normality of Residuals:\\
Linearity:\\

\begin{center}\rule{0.5\linewidth}{0.5pt}\end{center}

\hypertarget{q9}{%
\subsection{Q9}\label{q9}}

Use model.matrix() to examine the design or model matrix (but you do not
need to include it in your assignment).\\
How many rows are there? How does the number of rows relate to the
number of observations (n)?\\
How many columns are there? How does the number of columns relate to the
number of model coefficients/parameters/''betas''?\\

\begin{center}\rule{0.5\linewidth}{0.5pt}\end{center}

Number of rows = ? =

Number of cols = ? =

\begin{center}\rule{0.5\linewidth}{0.5pt}\end{center}

\hypertarget{appendix}{%
\section{Appendix}\label{appendix}}

\begin{Shaded}
\begin{Highlighting}[]
\CommentTok{\#Retain this code chunk!!!}
\FunctionTok{library}\NormalTok{(knitr)}
\NormalTok{knitr}\SpecialCharTok{::}\NormalTok{opts\_chunk}\SpecialCharTok{$}\FunctionTok{set}\NormalTok{(}\AttributeTok{echo =} \ConstantTok{FALSE}\NormalTok{)}
\NormalTok{knitr}\SpecialCharTok{::}\NormalTok{opts\_chunk}\SpecialCharTok{$}\FunctionTok{set}\NormalTok{(}\AttributeTok{message =} \ConstantTok{FALSE}\NormalTok{)}
\CommentTok{\#Q1}
\FunctionTok{library}\NormalTok{(tidyverse)}

\CommentTok{\#Q2}


\CommentTok{\#Q3}
\FunctionTok{library}\NormalTok{(broom)}

\CommentTok{\#Q7 }

\CommentTok{\#Q9}
\end{Highlighting}
\end{Shaded}


\end{document}
