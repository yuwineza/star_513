% Options for packages loaded elsewhere
\PassOptionsToPackage{unicode}{hyperref}
\PassOptionsToPackage{hyphens}{url}
%
\documentclass[
]{article}
\usepackage{amsmath,amssymb}
\usepackage{iftex}
\ifPDFTeX
  \usepackage[T1]{fontenc}
  \usepackage[utf8]{inputenc}
  \usepackage{textcomp} % provide euro and other symbols
\else % if luatex or xetex
  \usepackage{unicode-math} % this also loads fontspec
  \defaultfontfeatures{Scale=MatchLowercase}
  \defaultfontfeatures[\rmfamily]{Ligatures=TeX,Scale=1}
\fi
\usepackage{lmodern}
\ifPDFTeX\else
  % xetex/luatex font selection
\fi
% Use upquote if available, for straight quotes in verbatim environments
\IfFileExists{upquote.sty}{\usepackage{upquote}}{}
\IfFileExists{microtype.sty}{% use microtype if available
  \usepackage[]{microtype}
  \UseMicrotypeSet[protrusion]{basicmath} % disable protrusion for tt fonts
}{}
\makeatletter
\@ifundefined{KOMAClassName}{% if non-KOMA class
  \IfFileExists{parskip.sty}{%
    \usepackage{parskip}
  }{% else
    \setlength{\parindent}{0pt}
    \setlength{\parskip}{6pt plus 2pt minus 1pt}}
}{% if KOMA class
  \KOMAoptions{parskip=half}}
\makeatother
\usepackage{xcolor}
\usepackage[margin=1in]{geometry}
\usepackage{color}
\usepackage{fancyvrb}
\newcommand{\VerbBar}{|}
\newcommand{\VERB}{\Verb[commandchars=\\\{\}]}
\DefineVerbatimEnvironment{Highlighting}{Verbatim}{commandchars=\\\{\}}
% Add ',fontsize=\small' for more characters per line
\usepackage{framed}
\definecolor{shadecolor}{RGB}{248,248,248}
\newenvironment{Shaded}{\begin{snugshade}}{\end{snugshade}}
\newcommand{\AlertTok}[1]{\textcolor[rgb]{0.94,0.16,0.16}{#1}}
\newcommand{\AnnotationTok}[1]{\textcolor[rgb]{0.56,0.35,0.01}{\textbf{\textit{#1}}}}
\newcommand{\AttributeTok}[1]{\textcolor[rgb]{0.13,0.29,0.53}{#1}}
\newcommand{\BaseNTok}[1]{\textcolor[rgb]{0.00,0.00,0.81}{#1}}
\newcommand{\BuiltInTok}[1]{#1}
\newcommand{\CharTok}[1]{\textcolor[rgb]{0.31,0.60,0.02}{#1}}
\newcommand{\CommentTok}[1]{\textcolor[rgb]{0.56,0.35,0.01}{\textit{#1}}}
\newcommand{\CommentVarTok}[1]{\textcolor[rgb]{0.56,0.35,0.01}{\textbf{\textit{#1}}}}
\newcommand{\ConstantTok}[1]{\textcolor[rgb]{0.56,0.35,0.01}{#1}}
\newcommand{\ControlFlowTok}[1]{\textcolor[rgb]{0.13,0.29,0.53}{\textbf{#1}}}
\newcommand{\DataTypeTok}[1]{\textcolor[rgb]{0.13,0.29,0.53}{#1}}
\newcommand{\DecValTok}[1]{\textcolor[rgb]{0.00,0.00,0.81}{#1}}
\newcommand{\DocumentationTok}[1]{\textcolor[rgb]{0.56,0.35,0.01}{\textbf{\textit{#1}}}}
\newcommand{\ErrorTok}[1]{\textcolor[rgb]{0.64,0.00,0.00}{\textbf{#1}}}
\newcommand{\ExtensionTok}[1]{#1}
\newcommand{\FloatTok}[1]{\textcolor[rgb]{0.00,0.00,0.81}{#1}}
\newcommand{\FunctionTok}[1]{\textcolor[rgb]{0.13,0.29,0.53}{\textbf{#1}}}
\newcommand{\ImportTok}[1]{#1}
\newcommand{\InformationTok}[1]{\textcolor[rgb]{0.56,0.35,0.01}{\textbf{\textit{#1}}}}
\newcommand{\KeywordTok}[1]{\textcolor[rgb]{0.13,0.29,0.53}{\textbf{#1}}}
\newcommand{\NormalTok}[1]{#1}
\newcommand{\OperatorTok}[1]{\textcolor[rgb]{0.81,0.36,0.00}{\textbf{#1}}}
\newcommand{\OtherTok}[1]{\textcolor[rgb]{0.56,0.35,0.01}{#1}}
\newcommand{\PreprocessorTok}[1]{\textcolor[rgb]{0.56,0.35,0.01}{\textit{#1}}}
\newcommand{\RegionMarkerTok}[1]{#1}
\newcommand{\SpecialCharTok}[1]{\textcolor[rgb]{0.81,0.36,0.00}{\textbf{#1}}}
\newcommand{\SpecialStringTok}[1]{\textcolor[rgb]{0.31,0.60,0.02}{#1}}
\newcommand{\StringTok}[1]{\textcolor[rgb]{0.31,0.60,0.02}{#1}}
\newcommand{\VariableTok}[1]{\textcolor[rgb]{0.00,0.00,0.00}{#1}}
\newcommand{\VerbatimStringTok}[1]{\textcolor[rgb]{0.31,0.60,0.02}{#1}}
\newcommand{\WarningTok}[1]{\textcolor[rgb]{0.56,0.35,0.01}{\textbf{\textit{#1}}}}
\usepackage{graphicx}
\makeatletter
\def\maxwidth{\ifdim\Gin@nat@width>\linewidth\linewidth\else\Gin@nat@width\fi}
\def\maxheight{\ifdim\Gin@nat@height>\textheight\textheight\else\Gin@nat@height\fi}
\makeatother
% Scale images if necessary, so that they will not overflow the page
% margins by default, and it is still possible to overwrite the defaults
% using explicit options in \includegraphics[width, height, ...]{}
\setkeys{Gin}{width=\maxwidth,height=\maxheight,keepaspectratio}
% Set default figure placement to htbp
\makeatletter
\def\fps@figure{htbp}
\makeatother
\setlength{\emergencystretch}{3em} % prevent overfull lines
\providecommand{\tightlist}{%
  \setlength{\itemsep}{0pt}\setlength{\parskip}{0pt}}
\setcounter{secnumdepth}{-\maxdimen} % remove section numbering
\ifLuaTeX
  \usepackage{selnolig}  % disable illegal ligatures
\fi
\IfFileExists{bookmark.sty}{\usepackage{bookmark}}{\usepackage{hyperref}}
\IfFileExists{xurl.sty}{\usepackage{xurl}}{} % add URL line breaks if available
\urlstyle{same}
\hypersetup{
  pdftitle={STAR 513: HW 3},
  pdfauthor={Yvette Uwineza},
  hidelinks,
  pdfcreator={LaTeX via pandoc}}

\title{STAR 513: HW 3}
\author{Yvette Uwineza}
\date{}

\begin{document}
\maketitle

Total points: 40\\
Questions are worth \textbf{2 pts} each, except where noted.\\
See Canvas calendar for due date.\\

Homework should be submitted as a pdf, doc or docx file via Canvas.\\
Use of R markdown HW template is strongly encouraged.\\
Add or delete code chunks as needed.\\
Knit frequently to avoid last minute problems!\\
Your submitted assignment should be neatly formatted and organized.\\

\textbf{Ott \& Longnecker Example 8.7:} It is conjectured that if fields
are overgrazed by cattle there will be soil compaction (which could lead
to reduced grass). A horticulturist at the agriculture experiment
station designed a study to evaluate the conjecture. Three grazing
regimens are considered:\\
- Continuous: continuous grazing\\
- Rest1week: three-week grazing then one-week no grazing and\\
- Rest2weeks: two-week grazing then two-weeks no grazing.\\
A total of 21 similar plots of land are selected for the study. Each of
the three grazing regimens are randomly assigned to 7 plots per regimen.
After the plots are subject to the grazing regimens for 4 mounts, the
the soil density (g/cm3) is measured for each plot.

The data GrazeData.csv is available from Canvas.

Prior to starting the statistical analysis, you will first need to
transpose the data to long format using code something like the
following. Modify the code as needed. For consistency, the levels of
graze should match the bullet list above. Be sure to check the modified
data!

\hypertarget{q1-4-pts}{%
\subsection{Q1 (4 pts)}\label{q1-4-pts}}

What is the predictor variable (x)? Is this categorical or numeric?\\
What is the response variable (y)? Is this categorical or numeric?

\begin{center}\rule{0.5\linewidth}{0.5pt}\end{center}

The predictor variable (x) is\\
The response variable (y) is

\begin{center}\rule{0.5\linewidth}{0.5pt}\end{center}

\hypertarget{q2-4-pts}{%
\subsection{Q2 (4 pts)}\label{q2-4-pts}}

Create a boxplot of the data. Your plot should include axis labels that
include the units where appropriate. Briefly comment on at least one
thing you learn from this plot.

\begin{center}\rule{0.5\linewidth}{0.5pt}\end{center}

Response

\begin{center}\rule{0.5\linewidth}{0.5pt}\end{center}

\hypertarget{q3-4-pts}{%
\subsection{Q3 (4 pts)}\label{q3-4-pts}}

Create a table of summary statistics of SoilDensity by Graze including
n, mean and standard deviation. Hint: Use tidyverse group\_by() and
summarise().

\begin{center}\rule{0.5\linewidth}{0.5pt}\end{center}

\begin{center}\rule{0.5\linewidth}{0.5pt}\end{center}

\hypertarget{q4}{%
\subsection{Q4}\label{q4}}

Fit an appropriate model and include the detailed ``coefficients table''
in your assignment. This table includes estimates, standard errors, test
statistics and p-values. This can be done using tidy() or summary().

\begin{center}\rule{0.5\linewidth}{0.5pt}\end{center}

\begin{center}\rule{0.5\linewidth}{0.5pt}\end{center}

\hypertarget{q5-3-pts}{%
\subsection{Q5 (3 pts)}\label{q5-3-pts}}

Calculate the estimated average soil density for each of grazing
regimens, using the coefficient (or parameter) estimates from the
previous question.\\
Notes:\\
(1) You must show your work to get full credit for this question.\\
(2) Use echo = TRUE to show your work for this question.\\
(3) Check your own work using simple means from Q3.

\begin{center}\rule{0.5\linewidth}{0.5pt}\end{center}

\begin{Shaded}
\begin{Highlighting}[]
\CommentTok{\#Q5}
\CommentTok{\#GrazeContinuous }

\CommentTok{\#Rest1week }

\CommentTok{\#Rest2weeks }
\end{Highlighting}
\end{Shaded}

\begin{center}\rule{0.5\linewidth}{0.5pt}\end{center}

\hypertarget{q6-3-pts}{%
\subsection{Q6 (3 pts)}\label{q6-3-pts}}

The estimate labeled ``Intercept'' is \(\hat\beta_0\). Provide a
detailed one-sentence interpretation of what is being estimated in
context of this study.

\begin{center}\rule{0.5\linewidth}{0.5pt}\end{center}

Response

\begin{center}\rule{0.5\linewidth}{0.5pt}\end{center}

\hypertarget{q7-4-pts}{%
\subsection{Q7 (4 pts)}\label{q7-4-pts}}

The estimate labeled ``GrazeRest1week'' is \(\hat\beta_1\). Provide a
detailed one-sentence interpretation of what is being estimated in
context of this study.

\begin{center}\rule{0.5\linewidth}{0.5pt}\end{center}

Response

\begin{center}\rule{0.5\linewidth}{0.5pt}\end{center}

\hypertarget{q8}{%
\subsection{Q8}\label{q8}}

Use model.matrix() to examine the design or model matrix (but you do not
need to include it in your assignment). Provide a brief description of
the second column labeled ``GrazeRest1week''.

\begin{center}\rule{0.5\linewidth}{0.5pt}\end{center}

Response

\begin{center}\rule{0.5\linewidth}{0.5pt}\end{center}

\hypertarget{q9}{%
\subsection{Q9}\label{q9}}

Which grazing regimen does R treat as the reference group? Why is this
the reference group?

\begin{center}\rule{0.5\linewidth}{0.5pt}\end{center}

Response

\begin{center}\rule{0.5\linewidth}{0.5pt}\end{center}

\hypertarget{q10}{%
\subsection{Q10}\label{q10}}

Provide an ANOVA table corresponding to the model.

\begin{center}\rule{0.5\linewidth}{0.5pt}\end{center}

\begin{center}\rule{0.5\linewidth}{0.5pt}\end{center}

\hypertarget{q11}{%
\subsection{Q11}\label{q11}}

The ANOVA F-test from the previous question corresponds to a null
hypothesis of \(H_0: \mu_1 = \mu_2 = \mu_3\). Provide a brief,
statistical conclusion using \(\alpha = 0.05\) (ex: Reject H0 or Fail to
Reject H0.)

\begin{center}\rule{0.5\linewidth}{0.5pt}\end{center}

Response

\begin{center}\rule{0.5\linewidth}{0.5pt}\end{center}

\hypertarget{q12}{%
\subsection{Q12}\label{q12}}

Now provide a conclusion in context of the research study, limiting
statistical jargon. Remember we don't accept H0!

\begin{center}\rule{0.5\linewidth}{0.5pt}\end{center}

Response

\begin{center}\rule{0.5\linewidth}{0.5pt}\end{center}

\hypertarget{q13}{%
\subsection{Q13}\label{q13}}

Use emmeans() to provide the emmeans (estimated marginal means).

\begin{center}\rule{0.5\linewidth}{0.5pt}\end{center}

\begin{center}\rule{0.5\linewidth}{0.5pt}\end{center}

\hypertarget{q14}{%
\subsection{Q14}\label{q14}}

Regardless of any previous results, use pairs(, adjust = ``none'') to
provide the unadjusted pairwise comparisons. Note: I ask for unadjusted
pairwise comparisons here for learning purposes. In most cases, we
prefer Tukey adjusted pairwise comparisons.

\begin{center}\rule{0.5\linewidth}{0.5pt}\end{center}

\begin{center}\rule{0.5\linewidth}{0.5pt}\end{center}

\hypertarget{q15}{%
\subsection{Q15}\label{q15}}

Which of the pairwise comparisons from Q14 already appeared in the
default output (from Q4)? Why do we get one additional comparison in
Q14?

\begin{center}\rule{0.5\linewidth}{0.5pt}\end{center}

Response

\begin{center}\rule{0.5\linewidth}{0.5pt}\end{center}

\hypertarget{appendix}{%
\section{Appendix}\label{appendix}}

\begin{Shaded}
\begin{Highlighting}[]
\CommentTok{\#Retain this code chunk!!!}
\NormalTok{knitr}\SpecialCharTok{::}\NormalTok{opts\_chunk}\SpecialCharTok{$}\FunctionTok{set}\NormalTok{(}\AttributeTok{echo =} \ConstantTok{FALSE}\NormalTok{)}
\NormalTok{knitr}\SpecialCharTok{::}\NormalTok{opts\_chunk}\SpecialCharTok{$}\FunctionTok{set}\NormalTok{(}\AttributeTok{message =} \ConstantTok{FALSE}\NormalTok{)}
\FunctionTok{library}\NormalTok{(knitr)}
\FunctionTok{library}\NormalTok{(tinytex)}
\CommentTok{\#Import and transpose}
\FunctionTok{library}\NormalTok{(tidyverse)}
\NormalTok{GrazeData }\OtherTok{\textless{}{-}} \FunctionTok{read.csv}\NormalTok{(}\StringTok{"GrazeData.csv"}\NormalTok{)}
\NormalTok{GrazeData }\OtherTok{\textless{}{-}}\NormalTok{ GrazeData }\SpecialCharTok{\%\textgreater{}\%}
  \FunctionTok{pivot\_longer}\NormalTok{(}\AttributeTok{cols =} \FunctionTok{everything}\NormalTok{(), }\AttributeTok{names\_to =} \StringTok{"Graze"}\NormalTok{, }\AttributeTok{values\_to =} \StringTok{"SoilDensity"}\NormalTok{) }\SpecialCharTok{\%\textgreater{}\%}
  \FunctionTok{mutate}\NormalTok{(}\AttributeTok{Graze =} \FunctionTok{as\_factor}\NormalTok{(Graze)) }\SpecialCharTok{\%\textgreater{}\%}
  \FunctionTok{arrange}\NormalTok{(Graze)}
\CommentTok{\#Q2}

\CommentTok{\#Q3}

\CommentTok{\#Q4}

\CommentTok{\#Q5}
\CommentTok{\#GrazeContinuous }

\CommentTok{\#Rest1week }

\CommentTok{\#Rest2weeks }


\CommentTok{\#Q8}

\CommentTok{\#Q10}

\CommentTok{\#Q13}

\CommentTok{\#Q14}
\end{Highlighting}
\end{Shaded}


\end{document}
